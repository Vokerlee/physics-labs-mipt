\documentclass[a4paper, 12pt]{article} % тип документа

%%%Библиотеки
    %\usepackage[warn]{mathtext}	
    \usepackage[T2A]{fontenc}   %Кодировка
    \usepackage[utf8]{inputenc} %Кодировка исходного текста
    \usepackage[english, russian]{babel} %Локализация и переносы
    \usepackage{caption}
    \usepackage{gensymb}
    %\usepackage{listings}
    \usepackage{amsmath, amsfonts, amssymb, amsthm, mathtools}
    %\usepackage[warn]{mathtext}
    %\usepackage[mathscr]{eucal}
    %\usepackage{wasysym}
    %\usepackage{graphicx} %Вставка картинок правильная
    %\usepackage{pgfplots}
    \usepackage{indentfirst}
    %\usepackage{float}    %Плавающие картинки
    %\usepackage{wrapfig}  %Обтекание фигур (таблиц, картинок и прочего)
    \usepackage{fancyhdr}  %Загрузим пакет
    %\usepackage{lscape}
    %\usepackage{xcolor}
    %\usepackage[normalem]{ulem}
    
    \usepackage{titlesec}
    \titlelabel{\thetitle.\quad}

    \usepackage{hyperref}

%%%Конец библиотек

%%%Настройка ссылок
    \hypersetup
    {
        colorlinks = true,
        linkcolor  = blue,
        filecolor  = magenta,
        urlcolor   = blue
    }
%%%Конец настройки ссылок


%%%Настройка колонтитулы
    \pagestyle{fancy}
    \fancyhead{}
    \fancyhead[L]{1.4.2}
    \fancyhead[R]{Глаз Роман, группа Б01-007}
    \fancyfoot[C]{\thepage}
%%%конец настройки колонтитулы



\begin{document}
                        %%%%Начало документа%%%%


%%%Начало титульника
\begin{titlepage}

    \newpage
    \begin{center}
        \normalsize Московский физико-технический институт \\(госудраственный университет)
    \end{center}

    \vspace{6em}

    \begin{center}
        \Large Лабораторная работа по общему курсу физики\\Механика
    \end{center}

    \vspace{1em}

    \begin{center}
        \Large \textbf{1.4.2. Оборотный маятник}
    \end{center}

    \vspace{2em}

    \begin{center}
        \large Глаз Роман Сергеевич \\
        Группа Б01-007
    \end{center}

    \vspace{\fill}

    \begin{center}
        Долгопрудный \\2021
    \end{center}
    
\end{titlepage}
%%%Конец Титульника



%%%Настройка оглавления и нумерации страниц
    \thispagestyle{empty}
    \newpage
    \tableofcontents
    \newpage
    \setcounter{page}{1}
%%%Настройка оглавления и нумерации страниц

\textbf{Цель работы:} определить величину ускорения свободного падения, пользуясь оборотным маятником.\\

\textbf{Используемое оборудование:} оборотный маятник, счётчик числа колебаний, секундомер, штангенциркуль с пределом измерений $1$ м.

\section{Теоретические сведения}

Так как Земля является неинерциальной системой отсчёта из-за её вращения, то полное ускорение свободного падения в какой-то точке должно учитывать это вращение.

Также следует учитывать, что Земля не является идеально круглым шаром и что вблизи поверхности Земли плотность может быть неравномерной, а значит и ускорение свободного падения локально может быть искажено.

Из-за сил притяжения Луны и Земли ускорение вободного падения также меняется, но вклад в данном случае незначительный.

Существует множество различных способов измерения ускореняи свободного падения. Мы данной работе мы для этого используем оборотный маятник.

Легко получить, что период малых колебаний физического маятника ищется по формуле
\[T = 2\pi \sqrt{\frac{I}{mga}}\]

Здесь $I$ -- момент инерции маятника, $m$ -- масса маятника, а $a$ -- расстояние до центра масс от оси колебания.

Массу маятника и период колебаний можно измерить с высокой точностью, но с моентом инерции возникают проблемы. Для решения этой проблемы мы и будем использовать оборотный маятник.

Метод оборотного маятника основан на том, что период колебаний не меняется, если не меняеятся приведённая длина маятника $\frac{I}{ma}$, значит можно смещать ось качания вдоль в точку, отстоящую от оси качаний на расстоние, равное приведённой длине маятника, и лежащую на одной прямой с точкой подвеса ицентром маятника.

Применяемый в настоящей оборотный маятник (см. рис) состоит из стальной пластины (стержня), на которой укреплены однородные призмы $\text{П}_1$ и $\text{П}_2$. Период колебаний можно изменять с помощью грузиков $\text{Г}_1$, $\text{Г}_2$ и $\text{Г}_3$.

Допустим, нам удалось найти такое положение грузов, при котором для периодов колебаний маятника выполняется
\[T_1 = T_2 = 2\pi \sqrt{\frac{I_1}{mgl_1}} = 2\pi \sqrt{\frac{I_2}{mgl_2}}\Rightarrow \frac{I_1}{l_1} = \frac{I_2}{l_2}\]

Здесь $I_1$ и $I_2$ -- моменты инерции маятника относительно призм $\text{П}_1$ и $\text{П}_2$. По теореме Гюйгенца-Штейнера они равны
\[I_1 = I_0 + ml_1^2,\text{ } I_2 = I_0 + ml^2_2\] 

Здесь $I_0$ -- момент инерции маятника относительно оси, проходящей через центр его масс. Из всех записанных выше уравнений можно получить, что
\[g = \frac{4\pi^2}{T^2}(l_1+l_2) = 4\pi^2\frac{L}{T^2}\]

Здесь $L$ -- расстояние между призмами, которое нужно померить с высокой точностью с помощью штангенциркуля. Заметим, что система уравнений полученное решение только для $l_1 \neq l_2$. 

На самом деле точного равенства периодов колебаний добиться сложно, поэтому имеет место слудующее:
\[T_1 = 2\pi \sqrt{\frac{I_1}{mgl_1}},\text{ }  T_2 = 2\pi \sqrt{\frac{I_2}{mgl_2}}\]
\[T_1^2gl_1 - T^2_2gl_2 = 4\pi^2(l_1^2 - l_2^2) \Rightarrow g = 4\pi^2 \frac{l_1^2 - l_2^2}{l_1 T_1^2 - l_2 T_2^2} = 4\pi^2\frac{L}{T_0^2} \Rightarrow\]
\[\Rightarrow T_0^2 = \frac{l_1 T_1^2 - l_2 T_2^2}{l_1 - l_2}\]

Для погрешности измерения ускорения свободного падения имеем:
\[\sigma_g = g \sqrt{\Big( \frac{\sigma_L}{L} \Big)^2 + \Big(2 \frac{\sigma_{T_0}}{T_0} \Big)^2}\]

Рассматрим, как найти $\sigma_{T_0}$. Найдём частные производные обоих частей ранее записанного выражения:
\[\frac{\partial T_0^2}{\partial T_1} = \frac{\partial \Big(\frac{l_1 T_1^2 - l_2 T_2^2}{l_1 - l_2} \Big)}{\partial T_1} \Rightarrow \frac{\partial T_0}{\partial T_1} = \frac{l_1}{l_1-l_2} \cdot \frac{T_1}{T_0}\]

\[\frac{\partial T_0^2}{\partial T_2} = \frac{\partial \Big(\frac{l_1 T_1^2 - l_2 T_2^2}{l_1 - l_2} \Big)}{\partial T_2} \Rightarrow \frac{\partial T_0}{\partial T_2} = -\frac{l_2}{l_1-l_2} \cdot \frac{T_2}{T_0}\]

Если разность $l_1-l_2$ мала, то частные производные, а значит и погрешности для периодов становятся высокими. Значит эти длины не должны быть схожими. 

Снова рассматрим выражение
\[T_0^2 = \frac{l_1 T_1^2 - l_2 T_2^2}{l_1 - l_2} = T^2_2 + \frac{l_1}{l_1-l_2}(T_1 + T_2)(T_2 - T_1)\]

Заметим, что если $l_1$ и $l_2$ различаются, то второе слагаемое в правой части равенства будет очень мало, так как $T_2 - T_1$ -- очень малая величина. Значит и в формуле для погрешностей эта поправка будет очень малой. Тогда её можно и не учитывать. Значит можно считать, что периоды колебаний равны $\sigma_{T_1} = \sigma_{T_2} = \sigma_{T}$ и их погрешности равны, тогда:

\[\sigma_{T_0} = \sqrt{\Big( \frac{\partial T_0}{\partial T_2} \sigma_{T_1} \Big)^2 + \Big( \frac{\partial T_0}{\partial T_2} \sigma_{T_2} \Big)^2} = \frac{\sqrt{l_1^2 + l_2^2}}{l_1 - l_2}\sigma_T\]
 
Также заметим, что расстояния $l_1$ и $l_2$ не должны отличаться на хоть какой-то порядок, так как получится, что какой-то из периодов колебаний получается большим и растёт сила трения, которую мы изначально не брали в рассмотрение.
\newpage

\section{Экспериментальная установка}

Схема установки уже была приведена на рисунке. 

Для определения числа колеьаний используется счётчик, состоящий из осветителя, фотоэлемента и пересчётного устройства. Лёгкий стержень, укеплённый на торце маятника, пересекает световой луч дважды. Значит если $n_1$ и $n_2$ -- значения на датчике до и после измерения, то количество колебаний равно $N = (n_2 - n_1)/2$, а период колебаний равен $T = t/N$, где $t$ -- время, измеряемое секундомером.

Для определения расстояний надо снять маятник с консоли, расположить горизонтально на специально подставке, имеющей острую грань. Перемещая маятник. находим его центр масс, откуда можно легко найти требуемые расстояния.\\

\section{Ход работы}

Для начала определим диапазон амплитуд, в пределах которого период колебаний маятника не зависит от амплитуды. Измерим время $100$ колебаний, установив некоторую начальную амплитуду. Далее уменьшим её в два раза и повторим измерения. Если времена окажутся практически одинаковыми, то в рассматренных диапазонах период колебаний не зависит от частоты. Если времена различаются прилично, то уменьшаем начальную амплитуду в два раза и повторяем всю процедуру заново.

Теперь рассматрим, как сильно влияют положения грузов оборотного матника на периоды колебаний. 

Поэтому для начала зафиксируем грузы $\text{Г}_1$ и $\text{Г}_2$ и будем менять положение третьего груза. Снимем зависимость периодов колебаний от положения этого груза:
\begin{center}
\begin{tabular}{|c|c|c|c|}
\hline 
$l$, дел. & $T_1$, с & $T_2$, с & $|T_2 - T_1|$, с \\ 
\hline 
0 & 1,426 & 1,420& 0,006 \\ 
\hline 
5 & 1,451 & 1,434& 0,017 \\ 
\hline 
10 & 1,476 & 1,448& 0,028 \\ 
\hline 
15 & 1,503 & 1,465& 0,038 \\ 
\hline 
\end{tabular}
\end{center} 

Теперь фиксируем грузы $\text{Г}_1$ и $\text{Г}_3$ и меняем положение второго груза, заносим всё в таблицу:

\begin{center}
\begin{tabular}{|c|c|c|c|}
\hline 
$l$, дел. & $T_1$, с & $T_2$, с & $|T_2 - T_1|$, с \\ 
\hline 
0 & 1,493 & 1,436 & 0,057 \\ 
\hline 
2 & 1,487 & 1,438 & 0,049 \\ 
\hline 
7 & 1,473 & 1,444 & 0,029 \\ 
\hline 
14 & 1,454 & 1,441 & 0,013 \\ 
\hline 
20 & 1,443 & 1,431 & 0,012 \\ 
\hline 

\end{tabular}
\end{center} 

Из таблицы видно, что на периоды $T_1$ и $T_2$ больше влияет второй груз, так как зависимости периодов от первого и второго грузов ведут себя примерно одинаково, так как мы имеем $2$ груза, один из которых в несколько раз меньше, при том, что мы сняли зависимость для тяжёлого груза, который влияет на момент инерции больше. 

Значит для того, чтобы получить одинаковые периоды колебаний, для начала нужно менять положение второго груза.

При некотором положении второго груза имеем периоды колебаний $T_1 = 1,420$ с, $T_2 = 1,424$ с.

Измерим длины $l_1$ и  $l_2$ и их отношение:
\[l_1 = 20,1 \text{ см, } l_2 = 29,7\text{ см, } \frac{l_1}{l_2} = 1,475\]

Таким образом, мы не совсем вошли в желательный предел отношения $1,5 \leqslant \frac{l_1}{l_2} \leqslant 3$. 

Теперь будем менять положения первого и третьего грузов так, чтобы разница периодов колебаний для двух положений маятника снижалась, при этом надо смещать боковые грузы так, чтобы центр масс смещался дальше от центра стержня, чтобы отношение увеличилось.

 Для некоторых положений получим достаточно хорошие значения для периодов: $T_1 = 1,4145$ с, $T_2 = 1,4145$ с. Более точные значения получить невозможно, так как точность измерений ограничены ценой деления секундомера.

Измерим теперь новое отношение длин:
\[l_1 = 19,9 \text{ см, } l_2 = 29,9\text{ см, } \frac{l_1}{l_2} = 1,5\]

Теперь отношение немного больше, чем было.

Измерим штангенциркулем длину между призмами: $L = 49,76$ см. Повторим измерения несколько раз: значение длины такое же. Погрешность штангенциркуля равна $\Delta L = 0,05$ мм.

Теперь измерим $300$ колебаний в выбранном положении грузов. Полное время равно $T_{\text{полн}} = 424,30$ с. Значит время одного колебания равно $T = 1,41432$ с. Для погрешностей имеем следующее:
\[ \Delta T_{\text{полн}} = 0,01 \text{ с, } \Delta T = 3,33 \cdot 10^{-5} \text{ с.} \]

Теперь можно посчитать ускорение свободного падения и его погрешность:
\[g = 4\pi^2\frac{L}{T^2} = 9,82075 \text{ м/c}^2\cong 9,821 \text{ м/c}^2\]
\[\Delta g = g \sqrt{\Big( \sigma_{L} \Big)^2 + \Big( 2\frac{\sqrt{l_1^2 + l_2^2}}{l_1 - l_2}\sigma_T \Big)^2} = \]
\[= g \sqrt{\Big( \frac{\Delta L} {L} \Big)^2 + \Big( 2\frac{\sqrt{l_1^2 + l_2^2}}{l_1 - l_2}\frac{\Delta T} {T}\Big)^2} = 0,01 \text{ м/c}^2\]

Относительная погрешность равна
\[\varepsilon = \frac{\Delta g}{g} = 0,1\%\]

\section{Выводы}

Заметим, что величина ускорения свободного падения измерена достаточно точно, так как $\varepsilon = 0,1\%$, при том абсолютная погрешность равна $0,01 \text{ м/c}^2$. То есть диапазон значений покрывает действительное значение ускорение свободного падения, равного примерно $9,815 \text{ м/c}^2$ в нашей местности. 



\end{document}